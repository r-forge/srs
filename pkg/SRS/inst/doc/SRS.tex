% -*- mode: noweb; noweb-default-code-mode: R-mode; -*-
\documentclass[11pt, letter]{article}

%%%
%%% For future reference
%%%\documentclass[8pt]{article}
%%%\usepackage[landscape]{typearea}
%%% Set my margins
%%%\setlength{\oddsidemargin}{0.0truein}
%%%\setlength{\evensidemargin}{0.0truein}
%%%\setlength{\textwidth}{9truein}
%%%\setlength{\topmargin}{0.0truein}
%%%\setlength{\textheight}{6.5truein}
%%%\setlength{\headsep}{0.0truein}
%%%\setlength{\headheight}{0.0truein}
%%%\setlength{\topskip}{0pt}
%%%

%% Set my margins

\setlength{\oddsidemargin}{0.0truein}
\setlength{\evensidemargin}{0.0truein}
\setlength{\textwidth}{6.5truein}
\setlength{\topmargin}{0.0truein}
\setlength{\textheight}{9.0truein}
\setlength{\headsep}{0.0truein}
\setlength{\headheight}{0.0truein}
\setlength{\topskip}{0pt}
%% End of margins

%%\pagestyle{myheadings}
%%\markboth{$Date$\hfil$Revision$}{\thepage}

%\VignetteIndexEntry{Subject Randomization System}
\usepackage{amsmath, amsfonts, amssymb}

\newcommand\calF{\mathcal{F}}
\newcommand\calC{\mathcal{C}}
\newcommand\yidotbar{\bar{y}_{i.}^{(s)}}
\newcommand\ydotdotbar{\bar{y}_{..}^{(s)}}
\newcommand\sidotbar{\bar{S}_{i.}^{(s)}}
\newcommand\sdotdotbar{\bar{S}_{..}^{(s)}}
\newcommand\yij{y_{ij}^{(s)}}
\newcommand\eij{e_{ij}^{(s)}}
\newcommand\tij{t_{ij}^{(s)}}

\numberwithin{equation}{section}
%%\renewcommand{\textfraction}{0.0}
%%\renewcommand{\topfraction}{0.9}
%%\renewcommand{\bottomfraction}{0.9}
%%\renewcommand{\floatpagefraction}{0.9}
\usepackage{float}
\usepackage{rotating}

\title{SRS: A Subject Randomization System}
\author{Balasubramanian Narasimhan}

\date{$Revision$ of $Date$}



\usepackage{Sweave}
\begin{document}

\maketitle
\tableofcontents

\section{Introduction}
\label{sec:intro}

SRS is a Subject Randomization System based on the paper by Pocock and
Simon (\cite{ps:1975}. It follows the development in the paper rather
closely. In this vignette we show how one might use the system in
designing and implementing randomizations for clinical trials. 

This vignette has two parts to it. The first part goes into detail
discussing some of the innards of the package. This is most meaningful
to those in our Biostatistics core who may recommend this software for
use in trials.  The second part is more of a HOWTO for conducting a
trial.

This package is written using S4 classes. No deep knowledge of S4
classes is assumed in what follows.

To use the package, we first attach it.

\begin{Schunk}
\begin{Sinput}
> library(SRS)
\end{Sinput}
\end{Schunk}

\section{The Basic Classes}
\label{sec:basic-classes}
There are two main classes that most users of the package will use:
\texttt{ClinicalExperiment} and \texttt{PocockSimonRandomizer}. The
class \texttt{ClinicalExperiment}, as the name implies, encapsulates
the characteristics of a clinical experiment. An instance of this
class is used to create an instance of the other class
\texttt{PocockSimonRandomizer} so that the randomizer remains
associated with a particular clinical experiment.

\section{A Simple Example}
\label{sec:simple-example}

\subsection{A Clinical Experiment}
\label{sec:clin-expt}

Let us create a simple clinical experiment object after invoking the
requisite package. The function \texttt{ClinicalExperiment} (as
distinct from the \texttt{ClinicalExperiment} class) is available for
us. 

\begin{Schunk}
\begin{Sinput}
> expt0 <- ClinicalExperiment(number.of.factors = 3, number.of.factor.levels = c(2, 
+     2, 3), number.of.treatments = 3)
\end{Sinput}
\end{Schunk}

This create an experiment with three factors and three treatments. The
first factor has 2 levels, the second 2, and the third 3. If none of
the arguments are specified, the default is to create a two-factor,
two-treatment experiment with each factor having two levels. One can
name the factors with the argument \texttt{factor.names} but default
names such as $F_1, F_2,\ldots$ are provided. The levels are currently
indicated by the suffixes \texttt{-1}, \texttt{-2}, etc., that are
attached to the factor names; a flexible naming scheme for this might
be introduced later.

It is useful to print the object to see what it contains.
\begin{Schunk}
\begin{Sinput}
> print(expt0)
\end{Sinput}
\begin{Soutput}
An object of class “ClinicalExperiment”
Slot "number.of.factors":
[1] 3

Slot "factor.names":
[1] "F1" "F2" "F3"

Slot "factor.level.names":
[[1]]
[1] "1" "2"

[[2]]
[1] "1" "2"

[[3]]
[1] "1" "2" "3"


Slot "number.of.factor.levels":
[1] 2 2 3

Slot "number.of.treatments":
[1] 3

Slot "treatment.names":
[1] "Tr1" "Tr2" "Tr3"
\end{Soutput}
\end{Schunk}

Of course, in anything other than a toy setting, one actually provides
some names for the factor and factor levels. We'll use this in what
follows. 

\begin{Schunk}
\begin{Sinput}
> expt <- ClinicalExperiment(number.of.factors = 3, factor.names = c("Sex", 
+     "Race", "Stage"), number.of.factor.levels = c(2, 2, 3), factor.level.names = list(c("Female", 
+     "Male"), c("Caucasian", "Non-caucasian"), c("I", "II", "III")), 
+     number.of.treatments = 3, treatment.names <- c("Placebo", 
+         "Arm1", "Arm2"))
> print(expt)
\end{Sinput}
\begin{Soutput}
An object of class “ClinicalExperiment”
Slot "number.of.factors":
[1] 3

Slot "factor.names":
[1] "Sex"   "Race"  "Stage"

Slot "factor.level.names":
[[1]]
[1] "Female" "Male"  

[[2]]
[1] "Caucasian"     "Non-caucasian"

[[3]]
[1] "I"   "II"  "III"


Slot "number.of.factor.levels":
[1] 2 2 3

Slot "number.of.treatments":
[1] 3

Slot "treatment.names":
[1] "Placebo" "Arm1"    "Arm2"   
\end{Soutput}
\end{Schunk}

\subsection{The PocockSimon Randomizer}
\label{sec:ps-randomizer}

Now let's create a randomizer that will work for this experiment.

\begin{Schunk}
\begin{Sinput}
> r.obj <- new("PocockSimonRandomizer", expt, as.integer(12345))
> print(r.obj)
\end{Sinput}
\begin{Soutput}
An object of class “PocockSimonRandomizer”
Slot "expt":
An object of class “ClinicalExperiment”
Slot "number.of.factors":
[1] 3

Slot "factor.names":
[1] "Sex"   "Race"  "Stage"

Slot "factor.level.names":
[[1]]
[1] "Female" "Male"  

[[2]]
[1] "Caucasian"     "Non-caucasian"

[[3]]
[1] "I"   "II"  "III"


Slot "number.of.factor.levels":
[1] 2 2 3

Slot "number.of.treatments":
[1] 3

Slot "treatment.names":
[1] "Placebo" "Arm1"    "Arm2"   


Slot "seed":
[1] 12345

Slot "stateTable":
        Sex:Female Sex:Male Race:Caucasian Race:Non-caucasian Stage:I Stage:II
Placebo          0        0              0                  0       0        0
Arm1             0        0              0                  0       0        0
Arm2             0        0              0                  0       0        0
        Stage:III
Placebo         0
Arm1            0
Arm2            0

Slot "tr.assignments":
data frame with 0 columns and 0 rows

Slot "tr.ratios":
[1] 0.3333333 0.3333333 0.3333333

Slot "d.func":
function (x) 
{
    diff(range(x))
}

Slot "g.func":
function (x) 
{
    sum(x)
}

Slot "p.func":
function (overallImbalance) 
{
    number.of.treatments <- length(overallImbalance)
    p.star <- 2/3
    k <- which(overallImbalance == min(overallImbalance))
    if (length(k) > 1) {
        k <- sample(k, 1)
    }
    p.vec <- rep((1 - p.star)/(number.of.treatments - 1), number.of.treatments)
    p.vec[k] <- p.star
    p.vec
}
\end{Soutput}
\end{Schunk}

Note that we don't have a helper constructor function (for no
particular reason) and so we had to use the \texttt{new} function to
create the object. (Indeed, that is what the
\texttt{ClinicalExperiment} function does behind the scenes.)
 
The output of the print above indicates that there are some default
settings for the randomizer. For example, the treatment ratios are all
1's indicating equal treatment preference; others such as \texttt{1 2
  1} could have been specified. Note the \texttt{stateTable} slot
which will summarize the margins of the factor distributions by
treatment. Since no randomization has been done, the slot
\texttt{tr.assignments} is empty. 

Of interest are the slots named \texttt{d.func}, \texttt{g.func} and
\texttt{p.func}. The \texttt{d.func} computes
imbalance due to assigning each of the treatments, \texttt{g.func}
computes the overall imbalance, and the \texttt{p.func} computes the
probabilities of assigning each treatment based on the overall
imbalance. All of these can be changed by the user. Default values for
these functions are the ones described in \cite{ps:1975}. 

\subsection{Using the Randomizer}
\label{sec:using}

Now that we have defined the experiment and the randomizer, we can
randomize several subjects using these classes. First some helper
functions that are useful in simulations.

\begin{Schunk}
\begin{Sinput}
> generateId <- function(i) {
+     if (i < 0 || i > 10000) {
+         stop("generateId: Arg expected to be between 1 and 9999")
+     }
+     zero.count <- 5 - trunc(log10(i)) - 1
+     prefix <- substring(10^zero.count, 2)
+     paste("ID.", prefix, i, sep = "")
+ }
> generateRandomFactors <- function(factor.levels) {
+     unlist(lapply(factor.levels, function(x) sample(x, 1)))
+ }
\end{Sinput}
\end{Schunk}

Now, we will run a 10 randomizations and print the results.

\begin{Schunk}
\begin{Sinput}
> for (i in 1:10) r.obj <- randomize(r.obj, generateId(i), generateRandomFactors(expt@factor.level.names))
> print(r.obj@tr.assignments)
\end{Sinput}
\begin{Soutput}
            Sex          Race Stage Treatment
ID.00001   Male Non-caucasian   III      Arm2
ID.00002 Female     Caucasian    II   Placebo
ID.00003 Female     Caucasian   III      Arm1
ID.00004 Female     Caucasian    II      Arm2
ID.00005 Female Non-caucasian    II      Arm1
ID.00006   Male Non-caucasian    II   Placebo
ID.00007   Male Non-caucasian     I      Arm1
ID.00008   Male     Caucasian     I      Arm2
ID.00009 Female Non-caucasian   III   Placebo
ID.00010 Female Non-caucasian    II   Placebo
\end{Soutput}
\end{Schunk}

Just in case we are only interested in the last assigned treatment:

\begin{Schunk}
\begin{Sinput}
> lastRandomization(r.obj)
\end{Sinput}
\begin{Soutput}
            Sex          Race Stage Treatment
ID.00010 Female Non-caucasian    II   Placebo
\end{Soutput}
\end{Schunk}

We can also look at the marginal distributions thus:

\begin{Schunk}
\begin{Sinput}
> print(r.obj@stateTable)
\end{Sinput}
\begin{Soutput}
        Sex:Female Sex:Male Race:Caucasian Race:Non-caucasian Stage:I Stage:II
Placebo          3        1              1                  3       0        3
Arm1             2        1              1                  2       1        1
Arm2             1        2              2                  1       1        1
        Stage:III
Placebo         1
Arm1            1
Arm2            1
\end{Soutput}
\end{Schunk}

\section{Customizing the Randomizer}
\label{sec:customization}

The functions for computing imbalance, overall imbalance and
probabilities can all be customized. These are best illustrated by
additional examples.

\subsection{A different imbalance function}
\label{sec:diff-imbalance}

Let's move away from the default range function to say the standard
deviation (\texttt{sd}) function.

\begin{Schunk}
\begin{Sinput}
> r.obj.2 <- new("PocockSimonRandomizer", expt, as.integer(12345), 
+     d.func = sd)
> print(r.obj.2@d.func)
\end{Sinput}
\begin{Soutput}
function (x, na.rm = FALSE) 
{
    if (is.matrix(x)) 
        apply(x, 2, sd, na.rm = na.rm)
    else if (is.vector(x)) 
        sqrt(var(x, na.rm = na.rm))
    else if (is.data.frame(x)) 
        sapply(x, sd, na.rm = na.rm)
    else sqrt(var(as.vector(x), na.rm = na.rm))
}
<environment: namespace:stats>
\end{Soutput}
\end{Schunk}

Now let's run that simulation again.
\begin{Schunk}
\begin{Sinput}
> for (i in 1:10) r.obj.2 <- randomize(r.obj.2, generateId(i), 
+     generateRandomFactors(expt@factor.level.names))
> print(r.obj.2@tr.assignments)
\end{Sinput}
\begin{Soutput}
            Sex          Race Stage Treatment
ID.00001   Male Non-caucasian   III      Arm2
ID.00002 Female     Caucasian    II   Placebo
ID.00003 Female     Caucasian   III      Arm1
ID.00004 Female     Caucasian    II      Arm2
ID.00005 Female Non-caucasian    II      Arm1
ID.00006   Male Non-caucasian    II   Placebo
ID.00007   Male Non-caucasian     I      Arm1
ID.00008   Male     Caucasian     I      Arm2
ID.00009 Female Non-caucasian   III   Placebo
ID.00010 Female Non-caucasian    II   Placebo
\end{Soutput}
\end{Schunk}

Now print the summaries.

\begin{Schunk}
\begin{Sinput}
> print(table(r.obj@tr.assignments[, "Treatment"]))
\end{Sinput}
\begin{Soutput}
   Arm1    Arm2 Placebo 
      3       3       4 
\end{Soutput}
\begin{Sinput}
> print(r.obj@stateTable)
\end{Sinput}
\begin{Soutput}
        Sex:Female Sex:Male Race:Caucasian Race:Non-caucasian Stage:I Stage:II
Placebo          3        1              1                  3       0        3
Arm1             2        1              1                  2       1        1
Arm2             1        2              2                  1       1        1
        Stage:III
Placebo         1
Arm1            1
Arm2            1
\end{Soutput}
\end{Schunk}

\subsection{Weighting factors differently}
\label{sec:weight-factors}

Now let's weight imbalance on factor 1 more than the others by a
factor of 5. We do this by modifying the \texttt{g.func}.

\begin{Schunk}
\begin{Sinput}
> g.func <- function(imbalances) {
+     factor.weights <- c(5, 1, 1)
+     imbalances %*% factor.weights
+ }
> r.obj.3 <- new("PocockSimonRandomizer", expt, as.integer(12345), 
+     d.func = sd, g.func = g.func)
> print(r.obj.3@g.func)
\end{Sinput}
\begin{Soutput}
function (imbalances) 
{
    factor.weights <- c(5, 1, 1)
    imbalances %*% factor.weights
}
\end{Soutput}
\end{Schunk}

Now the simulation.

\begin{Schunk}
\begin{Sinput}
> for (i in 1:1000) r.obj.3 <- randomize(r.obj.3, generateId(i), 
+     generateRandomFactors(expt@factor.level.names))
\end{Sinput}
\end{Schunk}

Let's look at the distribution of treatments and the marginal
distribution of factors.

\begin{Schunk}
\begin{Sinput}
> print(table(r.obj.3@tr.assignments[, "Treatment"]))
\end{Sinput}
\begin{Soutput}
   Arm1    Arm2 Placebo 
    335     333     332 
\end{Soutput}
\begin{Sinput}
> print(r.obj.3@stateTable)
\end{Sinput}
\begin{Soutput}
        Sex:Female Sex:Male Race:Caucasian Race:Non-caucasian Stage:I Stage:II
Placebo        167      165            158                174     115      103
Arm1           167      168            160                175     116      102
Arm2           167      166            159                174     116      103
        Stage:III
Placebo       114
Arm1          117
Arm2          114
\end{Soutput}
\end{Schunk}

\subsection{Unequal treatment assignments}
\label{sec:unequal-randomization}

Next, we try a simulation where we require 5:2:1 randomization. To
really see the effect, we need to change the function that computes
probabilities for picking each treatment based on the
randomization. Let's be greedy and use the following:

\begin{Schunk}
\begin{Sinput}
> p.func.greedy <- function(overallImbalance) {
+     number.of.treatments <- length(overallImbalance)
+     k <- which(overallImbalance == min(overallImbalance))
+     p.vec <- rep(0, number.of.treatments)
+     p.vec[k] <- 1
+     p.vec/sum(p.vec)
+ }
\end{Sinput}
\end{Schunk}

Now, a new randomizer.

\begin{Schunk}
\begin{Sinput}
> r.obj.4 <- new("PocockSimonRandomizer", expt, as.integer(12345), 
+     tr.ratios = c(5, 2, 1), p.func = p.func.greedy)
\end{Sinput}
\end{Schunk}

A simulation.

\begin{Schunk}
\begin{Sinput}
> for (i in 1:1000) r.obj.4 <- randomize(r.obj.4, generateId(i), 
+     generateRandomFactors(expt@factor.level.names))
\end{Sinput}
\end{Schunk}

\begin{Schunk}
\begin{Sinput}
> print(table(r.obj.4@tr.assignments[, "Treatment"]))
\end{Sinput}
\begin{Soutput}
   Arm1    Arm2 Placebo 
    250     125     625 
\end{Soutput}
\begin{Sinput}
> print(r.obj.4@stateTable)
\end{Sinput}
\begin{Soutput}
        Sex:Female Sex:Male Race:Caucasian Race:Non-caucasian Stage:I Stage:II
Placebo        312      313            309                316     206      202
Arm1           125      125            124                126      82       81
Arm2            62       63             62                 63      42       40
        Stage:III
Placebo       217
Arm1           87
Arm2           43
\end{Soutput}
\end{Schunk}

\subsection{A different probability assignment}
\label{sec:diff-prob}

The drawback of using the greedy function in the previous example is
that there is some predictability as to what the randomizer will
assign based on the current state.  To throw in a bit of uncertainty,
we can define another function that favors the appropriate treatment
heavily, but not deterministically.

\begin{Schunk}
\begin{Sinput}
> p.func.not.so.greedy <- function(overallImbalance) {
+     FAVORED.PROB <- 0.75
+     number.of.treatments <- length(overallImbalance)
+     k <- which(overallImbalance == min(overallImbalance))
+     if (length(k) > 1) {
+         k <- sample(k, 1)
+     }
+     p.vec <- rep((1 - FAVORED.PROB)/(number.of.treatments - 1), 
+         number.of.treatments)
+     p.vec[k] <- FAVORED.PROB
+     p.vec
+ }
\end{Sinput}
\end{Schunk}

\begin{Schunk}
\begin{Sinput}
> r.obj.5 <- new("PocockSimonRandomizer", expt, as.integer(12345), 
+     tr.ratios = c(5, 2, 1), p.func = p.func.not.so.greedy)
\end{Sinput}
\end{Schunk}

A simulation.

\begin{Schunk}
\begin{Sinput}
> for (i in 1:1000) r.obj.5 <- randomize(r.obj.5, generateId(i), 
+     generateRandomFactors(expt@factor.level.names))
\end{Sinput}
\end{Schunk}

\begin{Schunk}
\begin{Sinput}
> print(table(r.obj.5@tr.assignments[, "Treatment"]))
\end{Sinput}
\begin{Soutput}
   Arm1    Arm2 Placebo 
    249     131     620 
\end{Soutput}
\begin{Sinput}
> print(r.obj.5@stateTable)
\end{Sinput}
\begin{Soutput}
        Sex:Female Sex:Male Race:Caucasian Race:Non-caucasian Stage:I Stage:II
Placebo        301      319            295                325     210      192
Arm1           120      129            115                134      84       78
Arm2            64       67             71                 60      45       40
        Stage:III
Placebo       218
Arm1           87
Arm2           46
\end{Soutput}
\end{Schunk}

Another possibility for the probability function might be based on the
actual imbalances.

\begin{Schunk}
\begin{Sinput}
> p.func.imbalance <- function(overallImbalance) {
+     p.vec <- overallImbalance/sum(overallImbalance)
+     p.vec
+ }
\end{Sinput}
\end{Schunk}

Of course, this assumes that the imbalances calculated are
non-negative, which would be the case with range or standard
deviation. But some care must be taken to ensure this is the case for
arbitrary situations.

\section{Notes}
\label{sec:notes}

The current package can be used without recourse to a database for
persistence.  This would require the initial definition of the
clinical experiment as in the example(s) above along with the
randomizer. This is done once for a study on a designated computer
running R to which the person assigned to do the randomization will
have primary access. 

Thereafter, every time a subject is to be randomized (after all the
usual procedures for registration in the study) the randomization
process will require merely an id for the subject and the levels of
the prognostic factors of interest. The randomization is performed
simply by running the code snippet

\begin{verbatim}
r.obj <- randomize(r.obj, id, c(fac1, fac2, fac3))
lastRandomization(r.obj)
\end{verbatim}

where \texttt{r.obj} is a randomizer created as above, and
\texttt{id}, \texttt{fac1}, \texttt{fac2}, \texttt{fac3}, are the
study id and the associated factor levels of the subject to be
randomized. 

After each assignment, the person can save the R workspace so that the
state is preserved.  If R is invoked from the same directory again,
the state is restored for subsequent randomizations. Of course, this
means all the usual responsibilities for saving the workspace apply
for this mode of operation. 

\bibliographystyle{plain}

\bibliography{SRS}


\end{document}
